
%Titulo:
% Modelos para localizaci�n de ajustadores de accidentes automovil�sticos
%Abstract:
% En este trabajo se intentan establecer pol�ticas de ubicaci�n
% de ajustadores de accidentes automovil�sticos
% con el fin de atender a los usuarios en el menor tiempo posible.
% Se presentar�n modelos de programaci�n entera mixta, 
% en los cuales se considera, por un lado,
% la asignaci�n fija de bases a cada ajustador,
% y por otro, asignaciones din�micas y pol�ticas de re-despliegue.

\documentclass[10pt,usenames,dvipsnames,svgnames,table]{beamer}
\usetheme{Warsaw}
\usepackage[english]{babel}
\usepackage[latin1]{inputenc}

\usepackage{graphicx}
\usepackage{epstopdf}
\graphicspath{ {./imagenes/} }
\title{Models for locating car wreck adjusters}
\subtitle{ENOAN 2015}
\author{Luis Maltos}
\institute[Universidad Aut\'onoma de Nuevo Le\'on]{
  Posgrado en Ingenier\'ia de Sistemas \\
  FIME / UANL}

\date[ENOAN 2015]{September 2015}

\AtBeginSubsection[]
{
  \begin{frame}
    \frametitle{Table of Contents}
    \tableofcontents[currentsubsection]
  \end{frame}
}

\begin{document}
\begin{frame}
  \titlepage
\end{frame}

\begin{frame}{Table of Contents}
  \tableofcontents
\end{frame}


%En ella se deben exponer brevemente pero con absoluta claridad, 
% la novedad y actualidad del tema,
% el objeto de la investigaci�n,
% sus objetivos,
% la hip�tesis de trabajo,
% el fundamento metodol�gico y
% los m�todos utilizados para realizar el trabajo de investigaci�n.
%Es decir, que la introducci�n es la fundamentaci�n cient�fica de la tesis en forma resumida.

\section{Introduction}
\begin{frame}

  The aim of this thesis is to support decision making concerning to
  the location and redeployment of insurance agents to attend road accidents.

  The methods of solution that will be proposed to will be improve the service
  offered by insurance agents, helping them to reach in less time,
  or determine the number of adjusters required to perform the service within
  the desired standards.
\end{frame}

\subsection{Problem}
\begin{frame}

  \textit{Determine the optimal bases (locations)
  for adjusters available, 
  so that the average response time
  to claims incurred in a given region,
  whether in the shortest possible time}
\end{frame}

\subsection{Motivation}
\begin{frame}
  When a car accident occurs, 
  it begins to increase traffic congestion on surrounding roads.
  This because the presence of an adjuster (proficient)
  which record and determine the causes of the accident,
  in order to move the car from the accident area and restore the flow.
  \begin{center}
    \includegraphics[scale=0.25]{389200-G}
  \end{center}
\end{frame}

\subsection{Background}
\begin{frame}[allowframebreaks]
  Richard C. Larson (1974) \cite{larson1974hypercube} 
  proposes the hyper-cube model.

  James P. Jarvis (1985) \cite{jarvis1985approximating} incorporates
  location dependent service times characteristics,
  developing an approximation model for a spatially distributed queuing system
  under general service time assumptions.

  Berman, Larson et. al (1987) \cite{berman1987stochastic}
  formulates the stochastic queue p-median problem,
  and propose a heuristic approach for locating cooperative service facilities
  on a network.

  Goldberg et. al (1990) \cite{goldberg1990validating}
  proposes a nonlinear integer programming
  model for finding optimal base locations for emergency medical vehicles.

\end{frame}


\section{Models}
% Validating and applying a model for locating emergency medical vehicles in Tucson, AZ
% Goldberg 1990

\subsection{Model A}

\begin{frame}
  This model is based on Goldberg's model, and it contains assignments
  variables for all possible orders.

  The following assumptions are required in the model:
  \begin{itemize}
  \item The probability that an adjuster is busy is $\rho$
    and is unaffected by the state of the system
  \item There is a strict ordering of the bases preferred for each zone
    that does not depend on the current state of the system. 
  \item All calls are answered by an adjuster originating from its base,
    not en route back to the base
  \item Different call types may have different service times
    and may or may not be included in the system objective function
  \item The arrival of calls to the system follows a stationary distribution
  \item The model is presented using a 0-queue assumption.
  \end{itemize}
  
\end{frame}


\begin{frame}
  Parameters:
  \begin{itemize}
  \item $V$ the set of demand points, with $|V| = n$
  \item $W$ the set of possible site for adjusters/facilities, with $|W| = m$
  \item $p$ number of response adjusters
  \item $\rho$ is the utilization of each vehicle
  \item $t_{ij}$ is the expected travel time between zone \textit{i} and base \textit{j}.
  \item $h_{ij}^{k}$ is the probability that $j$ attend client $i$ given that
    is the kith preferred.
  \end{itemize}
  
  Variables:
  \begin{itemize}
  \item $x_j$ it there is and adjuster in node $j$
  \item $y_{ij}^k$ if the facility j, is the k-th to cover the node $i$
  \end{itemize}
\end{frame}

\begin{frame}[allowframebreaks]{}{}

{\small
  \begin{equation}
    \min \, \sum_{j=1}^{m}{\sum_{\ell=1}^{k}{\sum_{i=1}^{n}{h_{ij}^{\ell}t_{ij}y_{ij}^{\ell}}}}
  \end{equation}
}
{\small
  \begin{align}
    \sum_{j \in W}{x_j} & = p               &                                  &\\
    \sum_{j \in W}{y_{ij}^{k}} & = 1        &         \forall i \in V, k &\in I \\
    y_{ij}^{k} & \leq x_j                   & \forall i \in V,j \in W, k &\in I \\
    \sum_{k = 1}^{p}{y_{ij}^{k}} & \leq x_j &         \forall i \in V, j &\in W \\
    y_{ij}^{k} &\leq \sum_{r}{y_{ir}^{k-1}} & \forall i \in V,j \in W, k &\in I/{1} \\
    x_{j} & \in \{0,1\}      &                 \forall j &\in W \nonumber\\
    y_{ij}^{k} & \in \{0,1\} & \forall i \in V,j \in W,k &\in I \nonumber
  \end{align}
}
\end{frame}

% A model for the Stochastic Queue p-Median Problem
% Created by Luis Maltos & Roger Rios
% 2015

\subsection{Model B}
  
\begin{frame}
  This model was developed with the idea that is unlikely that
  the furthest adjusters attend a client on cases where the system does not become congested.
  In that cases we can make the assumption that the probability of being served by the $\ell$th 
  adjuster is almost zero, when $\ell$ is large enough but less than $p$.

  Parametros:
  \begin{itemize}
  \item $M$ is a large integer.
  \item $\ell$ the number of response units of importance
  \item $a_{ik}$ kth location server regarding the node $i$
  \end{itemize}

  Variables:
  \begin{itemize}
  \item $z_j$ the number of adjusters in node $j$
  \item $y_{ij}^k$ if the facility j, is the k-th to cover the node $i$
  \end{itemize}

\end{frame}

\begin{frame}[allowframebreaks]
  The objective, and the equations from (2)-(5) they are practically the same,
  with the difference that the variable $x_j$ of the Model A
  was changed to integer variable $z_j$ inspired by the results of Berman,
  plus we had to add additional variables.
  {\small
    \begin{itemize}
    \item $u_{ij} \begin{cases} 1 & \mbox{if the number of adjusters between } i,j \mbox{ inclusive are less than } \ell \\
      0 & \mbox{otherwise}
    \end{cases}$
    \item $v_{ij}\begin{cases} 1 & \mbox{if the number of adjusters between } i,j \mbox{ inclusive are less than } \ell - 1 \\
      0 & \mbox{otherwise}
    \end{cases}$
    \end{itemize}
  }

{\footnotesize
  \begin{align}
    \sum_{r = 1}^{k \mid a_{ik}=j}{z_{a_{ir}}} + (p-\ell) u_{ij} & \leq p & \forall i \in V, j &\in W \\
    \sum_{r = 1}^{k \mid a_{ik}=j}{z_{a_{ir}}} + M u_{ij} & \geq \ell+1   & \forall i \in V, j &\in W \\
    \sum_{k = 1}^{\ell}{y_{ij}^{k}} + M (1 - u_{ij}) & \geq z_j           & \forall i \in V, j &\in W
  \end{align}
  
  \begin{align}
    \sum_{r = 1}^{k \mid a_{i(k+1)}=j}{z_{a_{ir}}} + (p-(\ell-1)) v_{ij} & \leq p                                     & \forall i \in V, j &\in W\\
    \sum_{r = 1}^{k \mid a_{i(k+1)}=j}{z_{a_{ir}}} + M v_{ij}         & \geq \ell                                     & \forall i \in V, j &\in W\\
    \sum_{k=1}^{k}{y_{ij}^{k}} + M (1 - v_{ij} + u_{ij}) & \geq \ell - \sum_{r = 1}^{k \mid a_{i(k+1)}=j}{z_{a_{ir}}} & \forall i \in V, j &\in W\\
    \sum_{k=1}^{k}{y_{ij}^{k}} - M (1 - v_{ij} + u_{ij}) & \leq \ell - \sum_{r = 1}^{k \mid a_{i(k+1)}=j}{z_{a_{ir}}} & \forall i \in V, j &\in W\\
    y_{ij}^{k} & \leq u_{ij} + v_{ij}  &       \forall i \in V,j &\in W \\
    z_j & \in \{0,1,\ldots,p\}         &               \forall j &\in V \nonumber\\
    y_{ij}^{k} & \in \{0,1\}           & \forall i\in V,j\in W,k &\in I \nonumber\\
    u_{ij},v_{ij} & \in \{0,1\}        &       \forall i \in V,j &\in W \nonumber
  \end{align}
}
\end{frame}


%\include{Experimentation}

\subsection{Experimentation}
\begin{frame}
  
  Different random instances were created,
  Some results are shown graphically
  \begin{center}
    \includegraphics[scale=0.25]{grafica_01}
    \includegraphics[scale=0.25]{grafica_02}
  \end{center}
\end{frame}

\begin{frame}

  The times for some values of \textit{l} in relation with \textit{p} 
  increase drastically
  \begin{center}
    \includegraphics[scale=0.25]{grafica_02}
    \includegraphics[scale=0.25]{grafica_03}
  \end{center}
  
\end{frame}

\begin{frame}

  But the results for small values of \textit{l},
  they remain in effect for the rest of the instances.
  \begin{center}
    \includegraphics[scale=0.25]{grafica_04}
    \includegraphics[scale=0.25]{grafica_05}
  \end{center}
  
\end{frame}

\frame{\begin{figure}[h!]\centering{\includegraphics[scale=0.35]{Test_100_30_12_01}\caption{n = 100, m = 30, p = 12, l = 1}}\end{figure}}
\frame{\begin{figure}[h!]\centering{\includegraphics[scale=0.35]{Test_100_30_12_02}\caption{n = 100, m = 30, p = 12, l = 2}}\end{figure}}
\frame{\begin{figure}[h!]\centering{\includegraphics[scale=0.35]{Test_100_30_12_03}\caption{n = 100, m = 30, p = 12, l = 3}}\end{figure}}
\frame{\begin{figure}[h!]\centering{\includegraphics[scale=0.35]{Test_100_30_12_04}\caption{n = 100, m = 30, p = 12, l = 4}}\end{figure}}
\frame{\begin{figure}[h!]\centering{\includegraphics[scale=0.35]{Test_100_30_12_05}\caption{n = 100, m = 30, p = 12, l = 5}}\end{figure}}
\frame{\begin{figure}[h!]\centering{\includegraphics[scale=0.35]{Test_100_30_12_06}\caption{n = 100, m = 30, p = 12, l = 6}}\end{figure}}
\frame{\begin{figure}[h!]\centering{\includegraphics[scale=0.35]{Test_100_30_12_07}\caption{n = 100, m = 30, p = 12, l = 7}}\end{figure}}
\frame{\begin{figure}[h!]\centering{\includegraphics[scale=0.35]{Test_100_30_12_08}\caption{n = 100, m = 30, p = 12, l = 8}}\end{figure}}
\frame{\begin{figure}[h!]\centering{\includegraphics[scale=0.35]{Test_100_30_12_09}\caption{n = 100, m = 30, p = 12, l = 9}}\end{figure}}
\frame{\begin{figure}[h!]\centering{\includegraphics[scale=0.35]{Test_100_30_12_10}\caption{n = 100, m = 30, p = 12, l = 10}}\end{figure}}
\frame{\begin{figure}[h!]\centering{\includegraphics[scale=0.35]{Test_100_30_12_11}\caption{n = 100, m = 30, p = 12, l = 11}}\end{figure}}
\frame{\begin{figure}[h!]\centering{\includegraphics[scale=0.35]{Test_100_30_12_12}\caption{n = 100, m = 30, p = 12, l = 12}}\end{figure}}


%\include{Futre Wortk}

\subsection{Acknowledgments}
\begin{frame}{Acknowledgments}
\begin{itemize}
\item CONACYT (Graduate Fellowship)
\item CONACYT (grant 2011-1-166397)
\item UANL
\item FIME
\end{itemize}

\end{frame}


%%%

\begin{frame}[allowframebreaks]
  \frametitle<presentation>{Bibliografia}    
{\tiny
  \begin{thebibliography}{10}    
    \beamertemplatebookbibitems

  \bibitem{larson1974hypercube}
    Larson, Richard C,
    \emph{A hypercube queuing model for facility location and redistricting in urban emergency services},
    Computers \& Operations Research, volume 1, number 1, pages 67--95, Elsevier, 1974.
      
  \bibitem{jarvis1985approximating}
    Jarvis, James P,
    \emph{Approximating the equilibrium behavior of multi-server loss systems},
    Management Science, volume 31, number 2, pages 235--239, INFORMS, 1985.

  \bibitem{berman1987stochastic}
    Berman, Oded and Larson, RC and Parkan, Celik,
    \emph{The stochastic queue p-median problem},
    Transportation Science, volume 21, no. 3, pages 207--216, INFORMS,1987.

  \bibitem{goldberg1990validating}
    Goldberg, Jeffrey and Dietrich, Robert and Chen, Jen Ming and Mitwasi, M George and Valenzuela, Terry and Criss, Elizabeth,
    \emph{Validating and applying a model for locating emergency medical vehicles in Tuczon, AZ},
    European Journal of Operational Research, volume 49, no. 3, pages 308--324, Elsevier, 1990.

%  \bibitem{marianov1998probabilistic}
%    Marianov, Vladimir and Serra, Daniel,
%    \emph{Probabilistic maximal covering location-allocation models for congested systems}
%    Journal of Regional Science, volume 38, number 3, pages 401--424, Wiley Online Library,1998.

%  \bibitem{rajagopalan2008multiperiod}
%    Rajagopalan, Hari K and Saydam, Cem and Xiao, Jing,
%    \emph{A multiperiod set covering location model for dynamic redeployment of ambulances},
%    Computers Operations Research, volume 35, no. 3, pages 814--826, Elsevier, 2008

%  \bibitem{pereira2015hybrid}
%    Pereira, Marcos A and Coelho, Leandro C and Lorena, Luiz AN and De Souza, Ligia C,
%    \emph{A hybrid method for the Probabilistic Maximal Covering Location--Allocation Problem},
%    Computers \& Operations Research, volume 57, pages 51--59, Elsevier, 2015.

  \end{thebibliography}
}
\end{frame}


\end{document}
